%!TEX root = ../coursework.tex

\section{Преобразование криптоаналитических задач (включая McEliece) в SAT, QUBO и модель Изинга}
Пусть дан открытый ключ в виде проверочной матрицы $\mathbf{H}\in\{0,1\}^{(n-k)\times n}$ и синдром $\mathbf{s}\in\{0,1\}^{(n-k)}$. Требуется найти вектор ошибки $\mathbf{e}\in\{0,1\}^n$ веса $\mathrm{wt}(\mathbf{e})=t$, удовлетворяющий
\[
\mathbf{H}\mathbf{e}^{\top}=\mathbf{s}^{\top}\quad(\bmod\ 2).
\]
При $\mathbf{s}=0$ это эквивалентно поиску кодового слова минимального веса в $\ker \mathbf{H}$ (bounded-distance decoding, BDD) \cite{ClassicMcEliece2020}.

\subsection{SAT/XOR-PB постановка}
Вводим булевы переменные $x_j\equiv e_j\in\{0,1\}$, $j=1,\dots,n$. Для строки $i$ определим множество индексов $N(i)=\{j\mid h_{ij}=1\}$. Тогда каждое уравнение порождает XOR-ограничение
\[
\bigoplus_{j\in N(i)} x_j = s_i .
\]
Эти ограничения выгодно хранить в \emph{родной XOR-форме} для классических решателей (гауссово исключение на лету), либо компилировать в CNF минимальными энкодингами XOR \cite{Zhang2000,Lafitte2014,Im2025}. Глобальное ограничение веса кодируем кардинальными схемами (sequential counter / totalizer / adder network), обеспечивающими $O(n\log n)$ дополнительных переменных и локальные клаузы. Практически: для точной постановки — «ровно $t$», для оптимизационной — «$\le t$» с мягкой затяжкой до равенства.

\subsection{От SAT/XOR-PB к QUBО: явная квадратизация}
Для квантового отжига требуется QUBO
\[
E(x)=\sum_i a_i x_i + \sum_{i<j} b_{ij}\,x_i x_j + \mathrm{const},\qquad x_i\in\{0,1\}.
\]
Ниже приводятся \textbf{квадратические} (2-локальные) штрафы с анциллами, гарантирующие корректность.

\paragraph{(a) 3-CNF клауза.}
Для дизъюнкта $(\ell_p\vee \ell_q\vee \ell_r)$, где литералы $\ell_\cdot\in\{0,1\}$ (негированный литерал $\bar x$ заменяем на $1-x$), штраф нарушения:
\[
E_{\mathrm{clause}}=A\,(1-\ell_p)(1-\ell_q)(1-\ell_r).
\]
Снижаем степень через анциллу $z\approx \ell_p\ell_q$:
\[
E_{\mathrm{clause}}^{(2)}=A\,(1-z)(1-\ell_r)\;+\;P_a\big(\ell_p\ell_q -2\ell_p z -2\ell_q z + 3 z\big).
\]
Последний QUBO-терм зануляется $\Leftrightarrow\ z=\ell_p\ell_q$ (для всех $0/1$-назначений), обеспечивая эквивалентность при $P_a$ достаточно большом.

\paragraph{(b) XOR-ограничение степени $d=|N(i)|$.}
Строим «параллельный» XOR из пар, вводя последовательность анцилл $y_{i,1},\dots,y_{i,d-1}$ и промежуточные переносы $t_{i,1},\dots,t_{i,d-1}$:
\[
\begin{aligned}
&y_{i,1}=x_{j_1}\oplus x_{j_2},\quad E^{(2)}_{i,1}=P_x\,(x_{j_1}+x_{j_2}+y_{i,1}-2t_{i,1})^2,\\
&y_{i,2}=y_{i,1}\oplus x_{j_3},\quad E^{(2)}_{i,2}=P_x\,(y_{i,1}+x_{j_3}+y_{i,2}-2t_{i,2})^2,\\
&\ \ \ \vdots\\
&y_{i,d-1}=y_{i,d-2}\oplus x_{j_d},\quad E^{(2)}_{i,d-1}=P_x\,(y_{i,d-2}+x_{j_d}+y_{i,d-1}-2t_{i,d-1})^2,\\
&\text{и, наконец,}\quad E^{(2)}_{i,\mathrm{par}}=P_p\,(y_{i,d-1}-s_i)^2.
\end{aligned}
\]
Каждый квадрат разворачивается в QUBO (только линейные и попарные члены); нуль энергии достигается тогда и только тогда, когда паритет выполним. Анцилл и переносов — $2(d-1)$ на уравнение (линейный оверхед по $d$).

\paragraph{(c) Ограничение веса.}
Плотная форма «мягкого равенства» удобна и \emph{корректна}:
\[
E_{\mathrm{wt}}=P_w\Big(\sum_{j=1}^n x_j - t\Big)^2
= P_w\Big(2\sum_{i<j}x_i x_j+(1-2t)\sum_j x_j\Big)+\mathrm{const}.
\]
Эта версия порождает плотные связи (all-to-all) между $x_i$; для аппаратной локальности допустимо заменить её «сумматором» (adder network / totalizer), сохранив квадратичность и разреженность \cite{Im2025}.

\paragraph{(d) Итоговая энергия.}
Полная QUBO-модель:
\[
E_{\text{QUBO}}(x)=\sum_{i}\bigg(\sum_{\alpha} E_{i,\alpha}^{(2)}\bigg)+\sum_{\text{клаузы }C}E^{(2)}_{\mathrm{clause}}(C)\;+\;E_{\mathrm{wt}}.
\]
Параметры выбираются иерархично: $P_a\gg P_x\gg P_p\gg P_w$ (см. ниже про масштабирование).

\subsection{Переход QUBO $\rightarrow$ Изинг}
Переходим к спинам $z_i\in\{-1,+1\}$ по $x_i=(1-z_i)/2$. Для
\[
E(x)=\sum_i a_i x_i + \sum_{i<j} b_{ij}\,x_i x_j + \mathrm{const}
\]
получаем гамильтониан Изинга
\[
H(z)=\sum_{i<j} J_{ij}\,z_i z_j + \sum_i h_i z_i + \mathrm{const}',
\]
где при симметричном $b_{ij}$ верны \emph{явные} формулы:
\[
J_{ij}=\frac{b_{ij}}{4},\qquad
h_i=-\frac{a_i}{2}-\frac{1}{4}\sum_{j\ne i} b_{ij}.
\]
Диагонали $b_{ii}$ можно поглотить в константу. Этот шаг нужен для запуска на отжигателях \cite{Bian2018,Sirdey2023,Pei2025}.

\subsection{Корректность штрафов и выбор масштабов}
Обозначим через $L_i=|h_i|+\sum_{j\ne i}|J_{ij}|$ локальную «связность по энергии» узла $i$, $L_{\max}=\max_i L_i$. Выбор штрафов обеспечивает, что \emph{любое} нарушение логики (XOR, связь анцилл, клауза) всегда «дороже» потенциального выигрыша за счёт прочих терминов. Достаточные условия:
\[
P_a \ge 4L_{\max},\qquad
P_x \ge 2L_{\max},\qquad
P_p \ge 2L_{\max},\qquad
P_w \in [0.5L_{\max},\,L_{\max}],
\]
после \emph{нормализации} коэффициентов к аппаратному диапазону. Интуитивно: для оптимума выгоднее удовлетворить логику (снять крупный штраф), чем «обманывать» за счёт малых $|h|,|J|$.

\subsection{Нормализация и квантование коэффициентов}
Реальные устройства ограничивают диапазоны и разрядность $h_i,J_{ij}$ (типично $|h_i|\le h_{\max}$, $|J_{ij}|\le J_{\max}$ и конечная дискретность) \cite{DWaveDocs}. Практический пайплайн:
\begin{enumerate}
  \item Соберите $a_i,b_{ij}$ (QUBO), затем пересчитайте $h_i,J_{ij}$ по формулам выше.
  \item Выберите общий масштаб $s>0$:
  \[
  s=\min\Big\{\frac{h_{\max}}{\max_i |h_i|},\ \frac{J_{\max}}{\max_{i<j}|J_{ij}|}\Big\}.
  \]
  Масштабируйте $h\leftarrow s\,h$, $J\leftarrow s\,J$, \emph{одновременно} умножив все штрафы $(P_a,P_x,P_p,P_w)$ на $s$.
  \item Квантование: округлите $(h,J)$ к доступной сетке устройства; при необходимости слегка увеличьте штрафы (на 1–2 кванта), чтобы компенсировать ошибку округления.
\end{enumerate}

\subsection{Встраивание и инженерные параметры отжига}
Из-за разреженной связности Chimera/Pegasus/Zephyr логическая переменная реализуется \emph{цепочкой} физических кубитов с сильными ферромагнитными связями. Ключевые практики:
\begin{itemize}
  \item \textbf{Локальность формулы.} Используйте кардинальные сети (totalizer/adder) и «деревья» для XOR, чтобы ограничить степень и длины цепочек \cite{Im2025}.
  \item \textbf{Алгоритмы встраивания.} Применяйте \textit{minorminer}, анализируйте среднюю длину цепочки и долю разрывов.
  \item \textbf{Сила цепочек.} Стартовая эвристика: $\lambda_{\text{chain}}\in[1.2,2.0]\cdot\max_{i} \sum_{j}|J_{ij}|$ на соответствующем куске эмбеддинга; затем тонкая настройка по метрике разрывов.
  \item \textbf{Gauge (spin-reversal) трансформации.} Усредните систематические смещения с 8–16 gauge-запусками.
  \item \textbf{Расписания.} Сканы по времени отжига, паузам и оффсетам; фиксируйте $(\text{\#runs},\ P_{\mathrm{succ}},\ \text{TTS})$.
\end{itemize}

\subsection{Процедура решения на квантовом отжиге}
\begin{enumerate}
  \item \textbf{Моделирование:} построить QUBO по (a)–(c), выбрать иерархию штрафов, пересчитать в Изинг.
  \item \textbf{Нормализация:} масштабировать и квантовать $(h,J)$ к допустимому диапазону, пересчитать $P_{\cdot}$.
  \item \textbf{Встраивание:} получить эмбеддинг (minorminer), подобрать $\lambda_{\text{chain}}$.
  \item \textbf{Отжиг:} выполнить серию запусков с сеткой параметров (время, паузы, gauges).
  \item \textbf{Декодирование:} \emph{unembed} (мажоритарка $\,+$ локальный поиск), проверить $\mathbf{H}e^\top=\mathbf{s}^\top$ и $\mathrm{wt}(e)=t$.
  \item \textbf{Итерации:} при необходимости подстроить $\{P_a,P_x,P_p,P_w,\lambda_{\text{chain}}\}$ и повторить.
\end{enumerate}

\subsection{Накладные расходы и ограничения масштабируемости}
Квадратизация XOR степени $d$ требует $O(d)$ анцилл, «ровно $t$» в плотной форме — $O(n^2)$ ребёр (или $O(n\log n)$ при сетевых схемах). На текущих топологиях число \emph{логических} переменных после встраивания существенно меньше числа физических кубитов; рост $n$ приводит к удлинению цепочек и падению $P_{\mathrm{succ}}$ \cite{Bian2018,Sirdey2023,Pei2025}. Практическая применимость на параметрах, близких к Classic McEliece, ограничена встраиваемостью и точностью коэффициентов; подход остаётся полезным для изучения сокращённых экземпляров и гибридных схем с последующей классической обработкой \cite{ClassicMcEliece2020,Im2025}.
