%!TEX root = ../coursework.tex
\phantomsection

\section{Введение}

Современный криптоанализ часто сводится к решению сложных комбинаторных задач; среди них ключевую роль играет задача булевой выполнимости (SAT — \emph{Boolean Satisfiability}). Задача SAT является первой известной NP-полной задачей и служит универсальным инструментом для формализации многих задач информатики и математики. При сведении криптоаналитических задач к SAT становится возможным применение мощных общих алгоритмов и промышленных SAT-решателей, которые активно развиваются и регулярно демонстрируют рост производительности. В частности, с помощью SAT-формализаций исследователям удавалось решать упрощённые варианты задач поиска ключей и поиска коллизий в хэш-функциях, а также анализировать симметричные шифры (см., например, \cite{Mironov2006,Lafitte2014,Zhang2000}). Следовательно, прогресс в области SAT-технологий напрямую влияет на возможности логического криптоанализа.

Классические SAT-алгоритмы — включая DPLL (Davis–Putnam-Logemann–Loveland) и современные варианты с обучением конфликтов CDCL (Conflict-Driven Clause Learning) — оснащены сильными эвристиками, однако в худших случаях их сложность остаётся экспоненциальной по размеру входа. Это ограничивает применимость прямого SAT-подхода к стойким криптосистемам, параметры которых специально подбираются так, чтобы соответствующие SAT-формулировки были практически нерешаемыми.

В последние годы активность исследователей привлекли квантовые методы вычислений. Некоторые ключевые квантовые алгоритмы демонстрируют асимптотические выигрыши: алгоритм Шора для факторизации и алгоритм Гровера для неструктурированного поиска являются классическими примерами \cite{Shor1994,Grover1996}. Отдельный интерес представляют квантовые отжигатели (quantum annealers) — специализированные устройства, реализующие оптимизацию в модели Изинга с использованием квантовых эффектов туннелирования и суперпозиции (в коммерческом исполнении — системы D-Wave). Идея применения квантового отжига в криптоанализе заключается в сведении криптографической задачи к задаче оптимизации (или SAT/Ising-формулировке), после чего отжигатель ищет глобальный минимум энерговой функции, потенциально обходя локальные минимумы эффективнее классических методов \cite{Bian2018}.

Цель данной работы — провести структурированный обзор и критический анализ применимости квантовых SAT-решателей и квантовых отжигателей (в частности, архитектуры D-Wave) для атак на кодовую криптосистему Мак-Элиса. Система Мак-Элиса (McEliece) базируется на сложности задачи декодирования случайного линейного кода — задаче NP-сложной природы, для которой на текущий момент не известны эффективные квантовые алгоритмы. Тем не менее остаются открытыми вопросы о практической применимости квантовых отжигателей и их способности обеспечить реальный выигрыш в криптоанализе \cite{ClassicMcEliece2020}.

\subsection*{Задачи исследования}
В работе решаются следующие задачи:
\begin{enumerate}
    \item проанализировать роль SAT-формализаций в логическом криптоанализе и примеры их успешного применения к классическим шифрам;
    \item описать конструкцию криптосистемы Мак-Элиса и формализовать её стойкость в виде задач, пригодных для SAT/Ising-редукции;
    \item рассмотреть архитектуру квантовых отжигателей (Chimera/Pegasus-графы, модель Изинга) и методы трансляции криптографических задач в формат, поддерживаемый этими устройствами;
    \item обобщить и критически оценить существующие исследования по использованию D-Wave и аналогичных систем в криптоанализе, включая ограничения и практические барьеры;
    \item выполнить (или при необходимости спланировать) оценку применимости квантовых SAT-методов к параметрам, близким к реальным значениям Мак-Элиса, и дать прогнозы по возможностям масштабирования.
\end{enumerate}

\subsection*{Научный вклад и ограничения}
Данная работа не претендует на доказательство поломки схемы Мак-Элиса; цель — аккуратно оценить имеющиеся подходы и сформулировать границы применимости квантовых отжигателей в контексте криптоанализа кодовых систем. В числе ожидаемых результатов — сводка ограничений текущих архитектур, рекомендации по преобразованию задач в модель Изинга и рекомендации по экспериментальной постановке задач для будущих исследований.

\subsection*{Структура работы}
Работа организована следующим образом. В разделе~2 даётся обзор роли SAT в криптоанализе и краткое описание основных SAT-алгоритмов. В разделе~3 подробно рассматривается криптосистема Мак-Элиса и формулируются соответствующие математические задачи. В разделе~4 описывается аппаратная и программная архитектура квантовых отжигателей (D-Wave) и методы редукции к модели Изинга. В разделе~5 представлены результаты обзора и (при наличии) экспериментальной оценки, обсуждаются ограничения и перспективы. Раздел~6 содержит выводы и рекомендации для дальнейших работ.

\phantomsection
