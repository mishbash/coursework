\section{Заключение}
Была проведена критическая оценка эффективности квантовых SAT-решателей для криптоанализа криптосистемы Мак-Элиса. На основании рассмотренных материалов можно заключить, что на текущем этапе развития квантовые решатели SAT (в частности, квантовые отжигатели) не представляют непосредственной угрозы для безопасности схемы Мак-Элиса. Эта кодовая криптосистема, являясь одним из основных представителей постквантовой криптографии, опирается на задачу декодирования случайного линейного кода, для которой квантовые алгоритмы не обеспечивают экспоненциального ускорения \cite{Grover1996}. Экспериментальные исследования показывают, что квантовый отжигатель способен решать лишь упрощённые версии задачи, при этом не демонстрируя существенного преимущества над классическими методами \cite{Bian2018,Sirdey2023,Pei2025}. Современные квантовые устройства обладают ограниченными возможностями и не позволяют обрабатывать задачи, соответствующие параметрам схемы Classic McEliece. Даже при гипотетическом масштабировании в ближайшей перспективе квантовые отжигатели вряд ли преодолеют экспоненциальный барьер: возможно, они смогут обрабатывать задачи немного более сложные, чем уже решаемые классическими алгоритмами \cite{Shintaro2024}, однако их вычислительная мощность останется недостаточной для атаки на реальные параметры \cite{ClassicMcEliece2020}.

При этом исследования квантовых SAT-решателей имеют конструктивное значение. Данный подход представляет собой новую парадигму и уже показал применимость к ряду задач оптимизации \cite{Bian2018,DWaveDocs}. В области криптоанализа преимущества пока наблюдаются лишь на малых примерах, однако дальнейшее развитие квантово-классических гибридных алгоритмов, а также улучшение кодировок задач может повысить эффективность данного подхода. Повышение мощности квантовых устройств (как отжигателей, так и вентильных компьютеров) также может расширить границы применимости квантового криптоанализа. При условии появления универсальных квантовых машин с тысячами безошибочных кубитов потребуется пересмотр общей оценки безопасности криптографических систем. В то же время криптосистема Мак-Элиса обладает значительным запасом прочности: даже с учётом квадратичных ускорений и ограниченных квантовых улучшений выбор параметров обеспечивает достаточный уровень стойкости, что отражено в стандартизованной версии Classic McEliece \cite{ClassicMcEliece2020}.

Таким образом, квантовые решатели SAT представляют собой перспективный инструмент, стимулирующий развитие как квантовых технологий, так и методов криптоанализа. Они позволяют реализовать атаки на некоторые задачи (SAT, MaxSAT, QUBO) в аппаратной форме с использованием принципов квантовой физики, однако пока не превосходят классические методы в применении к задачам, имеющим криптографическое значение \cite{Shintaro2024,Im2025}. В отношении постквантовых систем, включая McEliece, сохраняется устойчивая оценка безопасности: они выдерживают атаки как классических суперкомпьютеров, так и существующих квантовых устройств. Важно, однако, продолжать мониторинг прогресса в области квантовых вычислений и криптоанализа. Следует учитывать также экономический фактор: даже если квантовый отжигатель сможет в перспективе ускорить атаку на McEliece по сравнению с классическими алгоритмами, необходимо оценивать практическую значимость такого ускорения. Например, если время атаки сокращается с $10^{10}$ до $10^{6}$ лет, это не оказывает влияния на реальную стойкость.

\subsection*{Перспективы в контексте постквантовой криптостойкости}
Постквантовые алгоритмы проектируются с учётом возможного появления квантовых атак. Криптосистема Мак-Элиса зарекомендовала себя как устойчивая конструкция: большие размеры ключей компенсируются высокой степенью стойкости. Экспериментальные результаты показывают, что даже экзотические методы атаки (квантовые SAT-решатели, вариационные алгоритмы) не выявляют уязвимостей \cite{Pei2025,Sirdey2023}. Это повышает доверие к системе. Одновременно такие исследования позволяют уточнять запас прочности. В будущих версиях стандартов могут быть предложены параметры с увеличенными размерами кодов, если развитие квантовых отжигателей сделает это необходимым. Параллельно ведутся работы по созданию модификаций кодовых схем с использованием кодов низкой плотности проверок (LDPC codes) или смешанных метрик, которые могут оказаться более устойчивыми к квантовым атакам.

В заключение следует отметить, что квантовые SAT-решатели представляют интерес для криптоанализа, но не создают угрозы для криптосистемы Мак-Элиса в обозримой перспективе. Их преимущества ограничены специфическими сценариями, и отсутствуют свидетельства их применимости против реальных постквантовых параметров. Classic McEliece, наряду с другими финалистами стандартизации (CRYSTALS-Kyber, SPHINCS+ и др.), сохраняет статус надёжной криптосистемы против классических и квантовых атак \cite{ClassicMcEliece2020}. Продолжение исследований в этой области одновременно укрепляет существующие алгоритмы и способствует углублению понимания квантовых вычислений. Возможные результаты будущих исследований могут привести как к появлению более стойких схем, так и к развитию универсальных квантовых алгоритмов. В настоящий момент оценка остаётся неизменной: выбор алгоритмов и параметров в области постквантовой криптографии основывается на текущих научных данных и обеспечивает высокий уровень безопасности.
