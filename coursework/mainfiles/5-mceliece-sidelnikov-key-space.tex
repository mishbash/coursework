%!TEX root = ../coursework.tex

\section{Преобразование криптоаналитических задач (включая McEliece) в SAT и модель Изинга}
Дан открытый ключ в виде проверочной матрицы $\mathrm{H} \in\{0,1\}^{(n-k) \times n}$ и синдром $\mathrm{s} \in\{0,1\}^{(n-k)}$. Требуется найти вектор ошибки $\mathrm{e} \in\{0,1\}^{n}$ фиксированного веса $\operatorname{wt}(\mathrm{e})=t$, удовлетворяющий
$$
He^{\top}=s^{\top} \quad \text{над } F_2.
$$
При $s=0$ задача эквивалентна поиску кодового слова минимального веса в $\ker H$. Это и есть базовая форма декодирования на ограниченном расстоянии \cite{ClassicMcEliece2020}.

Вводим булевы переменные $x_j \equiv e_j$ для $j=1,\ldots,n$. Каждая строка $i$-я матрицы $H$ порождает XOR-ограничение
$$
\bigoplus_{j=1}^{n} h_{ij} x_j = s_i .
$$
Эти ограничения предпочтительно оставлять в XOR-родной форме (с последующим гауссовым исключением на лету) либо компилировать в CNF через минимальные энкодинги XOR \cite{Zhang2000,Lafitte2014,Im2025}. Глобальное ограничение $\mathrm{wt}(x)=t$ кодируется стандартными кардинальными энкодингами: sequential counter, totalizer, adder network (для равенства/неравенства), что влечёт $O(n \log n)$ вспомогательных переменных и клаузы с хорошей локальностью вывода. Альтернативы вида «каждые $t+1$ переменных не могут одновременно быть 1» ($n/(t+1)$ клауз) непрактичны уже при умеренных $n,t$.

Практика: для малых инстансов (игрушечные коды) используют точное «ровно $t$»; для оптимизационных постановок — «$\leq t$» с мягкой затяжкой до равенства.

Для запуска на квантовом отжигателе требуется квадратичная бинарная форма (QUBO) либо эквивалентный гамильтониан Изинга \cite{Bian2018,Sirdey2023,Pei2025}:
$$
H(z)=\sum_i h_i z_i + \sum_{(i,j)} J_{ij} z_i z_j, \quad z_i \in \{-1,+1\}.
$$
Переход $x \in \{0,1\} \leftrightarrow z \in \{-1,+1\}$ выполняется аффинно $x=(1-z)/2$.  

Для дизъюнкта $(\ell_p \vee \ell_q \vee \ell_r)$ вводится штраф
$$
E_{\text{clause}} = A \cdot (1-\ell_p)(1-\ell_q)(1-\ell_r),
$$
а затем понижается степень до квадратичной с помощью дополнительных переменных и квадратичных «сцепок» (классические методы кодирования с штрафными термами и вспомогательными переменными).

Для $\sum_j h_{ij} x_j \equiv s_i \pmod 2$ применяют малоглубинные схемы: либо прямые квадратичные штрафы из булевой алгебры, либо специализированные субсхемы, минимизирующие число анцилл \cite{Im2025}. Важно обеспечить баланс масштаба штрафов между XOR и остальными термами.

Ограничение $\sum_j x_j = t$ реализуют как квадратичную цель $(\sum_j x_j - t)^2$ с весом $B$, которая естественно ложится в QUBO. Такое «мягкое равенство» удобно для отжига: решения с $\operatorname{wt}(x) \neq t$ получают энергоштраф и вытесняются.

Коэффициенты $h_i, J_{ij}$ необходимо масштабировать в допустимый аппаратный диапазон (обычно $|h_i|,|J_{ij}| \leq 1$ при конечной дискретности). Выбор весов $A,B$ имеет принципиальное значение: если $\{A,B\}$ занижены, устройство будет «допускать» нарушения, что приведёт к ложным минимумам; если же завышены — утратится чувствительность к более тонкой структуре энергетического ландшафта. На практике применяют многошаговую калибровку с приведением коэффициентов к масштабу и использованием расписания отжига --- способом, по которому со временем изменяются параметры квантового отжига~\cite{DWaveDocs}.


Из-за разреженной связности топологий Chimera/Pegasus/Zephyr логическая переменная реализуется цепочкой физических кубитов, связанной сильными ферромагнитными $J$-термами (chain strength). Длина цепочек прямо снижает доступный логический размер и надёжность декодирования. Поэтому желательно:
\begin{itemize}
  \item \textbf{Проектирование кодирования с локальной связностью.} При построении QUBO/Изинг-формулировки желательно, чтобы большинство ограничений связывало только небольшие группы переменных. Это позволяет формировать компактные «кластеры» уравнений или подформул, которые проще отобразить на аппаратный граф связности квантового отжигателя. Такая кластеризация уменьшает длину цепочек при встраивании и снижает вероятность их разрыва.
  
  \item \textbf{Минимизация межкластерных связей.} Важно уменьшать количество рёбер, соединяющих разные кластеры, особенно если речь идёт о кардинальных ограничениях (например, условиях «ровно $t$»). Эти связи требуют длинных цепочек кубитов при встраивании, что приводит к росту ошибок и снижает эффективность отжига. Оптимизация структуры формулы с учётом топологии устройства повышает надёжность поиска решений.
  
  \item \textbf{Использование автоматических алгоритмов встраивания.} Для отображения логических переменных на физические кубиты применяются специализированные алгоритмы (например, \textit{minorminer}), которые автоматически подбирают конфигурацию цепочек. После автоматического встраивания необходима ручная настройка критических цепочек: корректировка силы связей внутри цепочек и анализ проблемных мест, где возможны разрывы. Такая комбинированная стратегия позволяет добиться баланса между масштабируемостью и надёжностью решения.
\end{itemize}

\subsection*{Процедура решения с использованием квантового отжига}
\begin{enumerate}
  \item \textbf{Моделирование.} На первом этапе задача сводится из SAT/XOR-PB формулировки к эквивалентной QUBO- или Изинг-модели. Ограничения переводятся в квадратичные штрафные функции, а коэффициенты тщательно калибруются так, чтобы нарушения условий получали достаточные энергетические штрафы, но при этом сохранялась чувствительность к структуре задачи.
  
  \item \textbf{Отображение (встраивание).} Логические переменные сопоставляются с физическими кубитами устройства. Поскольку топология квантового отжигателя имеет ограниченную связность, одна логическая переменная может реализовываться цепочкой физических кубитов, связанных сильными ферромагнитными взаимодействиями. На этом этапе настраивается сила связей внутри цепочек, чтобы минимизировать вероятность их разрыва.
  
  \item \textbf{Отжиг.} Выполняется серия запусков квантового отжига с различными параметрами: изменяются расписания эволюции и начальные состояния системы. В результате формируется множество (ансамбль) кандидатных решений, среди которых более вероятны низкоэнергетические конфигурации.
  
  \item \textbf{Декодирование и проверка.} После квантового этапа необходимо восстановить значения логических переменных из физических цепочек. Обычно используется мажоритарное правило, при котором выбирается наиболее часто встречающееся значение в цепочке. Затем проводится классическая проверка условий $He^T = s^T$ и $\mathrm{wt}(e) = t$, чтобы убедиться в корректности полученного решения.
  
  \item \textbf{Итерации.} Если решение не найдено или его качество неудовлетворительно, выполняется корректировка весов штрафов и параметров встраивания, после чего процесс отжига повторяется. Такой итеративный цикл позволяет улучшать результаты за счёт адаптации модели к особенностям конкретного экземпляра задачи.
\end{enumerate}


Ключевые узкие места при редукции синдромного декодирования к SAT/XOR-PB и далее к QUBO/Изингу для запуска на квантовом отжигателе сводятся к следующему. Во-первых, размерность: из-за разреженной связности аппаратных топологий (Chimera/Pegasus/Zephyr) и необходимости minor-встраивания из $N$ физических кубитов удаётся получить лишь сотни–единицы тысяч логических переменных при нетривиальной плотности рёбер; для реалистичных параметров McEliece это главный ограничитель. Во-вторых, чувствительность к параметрам: некорректный баланс штрафов ($A,B$) и силы цепочек приводит к ложным минимумам и разрывам цепей. В-третьих, отсутствие UNSAT-сертификации: квантовый отжиг — исключительно поисковая процедура; требуется классическая последующая классическая проверка (SAT/MaxSAT) \cite{Bian2018,Sirdey2023}.

Наконец, масштабные эффекты: для малых экземпляров задачи ($n$ порядка нескольких десятков, например $n=16$) редукция и встраивание фактически тривиальны, а производительность ограничивается самим поиском — классические решатели CDCL с поддержкой XOR-ограничений и квантовый отжиг (QA) справляются с такими задачами мгновенно. Переход к $n \sim 10^2$ уже требует значительных ресурсов, а $n \sim 10^3$ на текущих платформах остаётся недостижимым из-за дефицита связности и увеличения длины цепочек~\cite{Pei2025}.


В совокупности это определяет итоговый вывод: редукция «синдром + вес» к SAT/XOR-PB и далее к QUBO/Изингу технологически корректный и совместимый с QA путь, обеспечивающий адекватную постановку оптимизации, но на параметрах уровня Classic McEliece прямое применение квантового отжига нерелевантно из-за ограничений встраиваемости и параметрической чувствительности; практическая ценность подхода — калибровка, исследование сокращённых экземпляров задачи и гибридные схемы с последующей классической обработкой \cite{ClassicMcEliece2020,Im2025}.
