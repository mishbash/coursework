%!TEX root = ../coursework.tex

\section{Классические SAT-решатели против McEliece и квантовый отжиг DWave: постановка, редукции, пределы}



В логическом криптоанализе McEliece задача синдромного декодирования сводится к SAT/XOR-SAT: требуется найти вектор ошибки $e \in \{0,1\}^{n}$ веса $\mathrm{wt}(e)=t$, удовлетворяющий линейным ограничениям $He^{T}=s^{T}$ над $\mathrm{GF}(2)$. Практическая CNF содержит крупные XOR-блоки и кардинальные ограничения («ровно/не больше $t$»), реализованные с помощью сумматорных схем или тотализаторов~\cite{Zhang2000,Mironov2006,Lafitte2014}. Для полноразмерных параметров Classic McEliece (например, $n=3488$, $t=64$ и выше) наблюдается экспоненциальный комбинаторный рост: пространство $(n/t)$ становится колоссальным, матрицы Гоппы ведут себя квазислучайно (отсутствует «хребет» для CDCL), а массовые XOR-ограничения ухудшают локальность вывода. Даже решатели с поддержкой XOR (с гауссовым исключением «на лету») и гибридные SAT+XOR-ядра показывают выигрыш лишь на сокращённых экземплярах~\cite{Im2025}. В результате SAT оказывается применимым для калибровки и прототипирования на малых кодах, но не конкурирует с лучшими ISD-алгоритмами (Prange, Stern, Dumer, BJMM и их улучшениями) на стандартизованных параметрах: последние остаются экспоненциально сложными, но существенно эффективнее за счёт линейной алгебры и отбора информационного набора~\cite{Shintaro2024,ClassicMcEliece2020}.


\subsection*{Редукция SAT $\rightarrow$ QUBO(Quadratic Unconstrained Binary Optimization)/Изинг}
Для задействования квантовых оптимизаторов SAT/CNF преобразуют в квадратичную бинарную оптимизацию (QUBO) или эквивалентный гамильтониан Изинга \cite{Bian2018,Sirdey2023}:
$$
H(z)=\sum_i h_i z_i + \sum_{(i,j)} J_{ij}z_i z_j, \quad z_i \in\{-1,+1\},
$$
где клаузы/ограничения кодируются штрафами, а высокие степени понижаются введением вспомогательных переменных. XOR и кардинальность транслируются либо напрямую (через псевдобулевы штрафы), либо через специально оптимизированные схемы кодирования \cite{Pei2025}. Цель --- минимизировать $H$, что соответствует выполнению формулы/достижению целевого веса.

Аппаратный граф связности разрежен (Chimera/Pegasus/Zephyr), поэтому логические переменные требуют второстепенного встраивания: цепочки физических кубитов с сильной ферромагнитной связью реализуют переменные повышенной арности. Это «съедает» кубиты и ограничивает размер логической задачи (реалистично --- сотни/единицы тысяч логических переменных при умеренной плотности связей на системах $\sim 5$k физических кубитов) \cite{DWaveDocs}. Размещение --- отдельная задача, чувствительная к длинам цепочек и диапазону калибруемых коэффициентов ($h, J$).

Квантовый отжиг реализует адиабатическую эволюцию от простого гамильтониана к целевому $H$; квантовое туннелирование помогает преодолевать барьеры локальных минимумов. На уровне решателя это вероятностный поисковый метод: не доказывает UNSAT, а генерирует кандидаты решений с фокусом на низкоэнергетические конфигурации; на практике требуется множественный запуск и классическая валидация. Для крипто-SAT это приемлемо (решение заведомо существует), но выигрыш ограничен аппаратными и алгоритмическими факторами \cite{Sirdey2023,Pei2025}.

\subsection*{Факторы, ограничивающие квантовый прирост}
\begin{itemize}
  \item \textbf{Ограничения встраивания.} После отображения логических переменных в физические кубиты их количество резко сокращается: одна логическая переменная часто реализуется цепочкой физических кубитов. При этом плотные XOR-ограничения и кардинальные условия дополнительно ухудшают возможности встраивания. В результате реальный размер задачи, которую можно обработать на квантовом отжигателе, существенно меньше числа доступных кубитов.
  
  \item \textbf{Физические эффекты.} Конечная температура устройства, диссипация энергии и малые энергетические разрывы в финальной фазе эволюции приводят к тому, что система может «застыть» в локальных минимумах (субоптимальных решениях), не достигая глобального минимума.
  
  \item \textbf{Ограниченность модели.} Квантовый отжиг (QA) не выполняет логических выводов, характерных для классических SAT-решателей (например, CDCL или методов с гауссовым исключением), а лишь минимизирует функцию штрафов в энергетическом ландшафте. Поэтому корректность кодирования ограничений и правильный выбор масштаба коэффициентов являются критически важными для получения корректных решений.
  
  \item \textbf{Ограниченность наблюдаемых ускорений.} Экспериментальные выигрыши обычно фиксируются только на специально подобранных небольших примерах. Для кодовых задач McEliece экспоненциальная сложность сохраняется, и даже «константное» ускорение не компенсирует рост параметров при переходе к реальным криптографическим размерам~\cite{Bian2018,Pei2025}.
\end{itemize}


Классические SAT-решатели представляют собой инструмент для анализа малых экземпляров задачи и вспомогательных подзадач, но не создают угрозы для полноразмерной схемы; на практике наилучшие атаки обеспечиваются методами семейства ISD. Квантовый отжиг D-Wave позволяет рассматривать SAT/QUBO как задачу оптимизации в модели Изинга и даёт определённые ускорения на ограниченных классах задач, однако при реальных параметрах McEliece эффективность ограничивается проблемами встраивания, шумом и сохраняющейся экспоненциальной сложностью. Следовательно, ни классические SAT-решатели, ни их квантовая реализация посредством квантового отжига в настоящее время не подрывают стойкость схемы Classic McEliece. Дальнейший анализ представляется целесообразным в направлении гибридных подходов (например, использование ISD с квантово-ускоряемыми подзадачами) и разработки улучшенных методов кодирования и встраивания, но с учётом объективных ресурсных ограничений.
\cite{ClassicMcEliece2020,Shintaro2024}.
